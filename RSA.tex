
\documentclass[11pt,notitlepage]{report}
\textwidth 15cm 
\textheight 21.3cm
\evensidemargin 6mm
\oddsidemargin 6mm
\topmargin -1.1cm
\setlength{\parskip}{1.5ex}
\setcounter{secnumdepth}{1}
\usepackage{titlesec}
    \titleformat{\chapter}{\Large\centering}{}{0pt}{}{}
\usepackage{amsfonts,amsmath,amssymb,enumerate, amsthm}
\usepackage{enumitem}  
\usepackage{hyperref}
\counterwithout{section}{chapter}
\newcommand{\bb}[1]{\ensuremath{\mathbb{#1}}}
\newcommand{\tbf}[1]{\textbf{#1}}
\usepackage{minted}
\usemintedstyle{borland}
\usepackage[usenames,dvipsnames]{xcolor}
\usepackage{tcolorbox, tabularx, array, colortbl}
\tcbuselibrary{skins}
\tcbset{tab2/.style={enhanced,fonttitle=\bfseries,fontupper=\normalsize\sffamily,
colback=yellow!10!white,colframe=red!50!black,colbacktitle=Salmon!40!white,
coltitle=black,center title}}
\makeatletter
\newcommand*{\toccontents}{\@starttoc{toc}}

\begin{document}
\parindent=0pt
\title{\vspace{-15mm}Catalogue of RSA Attacks: From Various CTFs\vspace{-5mm}}
\author{by Sam Gunter}
\date{\small From: corCTF2021, CSAWCTF2021, PBJarCTF2021}
\maketitle
\toccontents

\thispagestyle{empty}
\newpage
\setcounter{page}{1}

\section{RSA Overview}

The RSA (Rivest-Shamir-Adleman) cryptosystem is dependant on the fact that some operations are easy to do in one direction, but very hard to do in the opposite direction. For example, multiplication is fast, factoring is slow.

Given primes $p_1, p_2 \dots$,\\
Calculate
$$n = p_1 p_2 \dots \text{ and } \lambda(n) = lcm(p_1-1, p_2-1, \dots)$$
with the least common multiple calculated by the Euclidean Algorithm

Now, choose an $e < \lambda(n)$ such that $e, \lambda(n)$ are coprime\\
Calculate
$$de \equiv 1 \pmod{\lambda(n)}$$
by the Extended Euclidean Algorithm with Bezouts Identity since $e, \lambda(n)$ are coprime\\


Now for a message $m$, calculate
$$m^e \equiv c \pmod{n}$$

Now, note that since $\lambda(n) = lcm(p_1-1, p_2-1, \dots)$ and $\lambda(n) = ed - 1$\\
By definition, for non-negative integers $k_1, k_2, k_3 \dots$
$$ed - 1 = k_1(p_2-1)(p_3-1)\dots = k_2(p_1-1)(p_2-1)\dots = \dots$$

For each $p_i$, if $m$ is a multiple of $p_i$ then 
$$m \equiv 0 \equiv (m^e)^d \pmod{p_i}$$
thus assume $m$ is not a multiple of $p_i$, then by Fermat's Little Theorem
$$m^{ed} \equiv m \left(m^{ed-1}\right) \equiv m \left(m^{k_j(p_1-1)\dots(p_i-1)\dots}\right) \equiv m \left(m^{(p_i - 1)}\right)^{k_j(p_1-1)\dots} \equiv m \left(1\right)^{k_j(p_1-1)\dots} \equiv m \pmod{p_i}$$
And since each $p_i$ is coprime, by the Chinese Remainder Theorem
$$m^{ed} \equiv m \pmod{n}$$

and thus
$$(c)^d \equiv (m^e)^d \equiv m \pmod{n}$$
as desired

\newpage
\subsection{Generating RSA Keys}

Typically, RSA is done with only 2 choices of primes, $p$ and $q$, as the difficulty in factoring comes from size, not the number of factors.

Also, $e$ is often the number $65537 = 2^{16} + 1$, as it is a prime this almost always satisfies that $e$ and $\lambda(n)$ are coprime, and raising numbers to this exponent is easier as it contains only one flipped bit.

\vspace{-4mm}
\begin{minted}{python}
# Returns d
def GenerateEPQ(e, p, q)

# Returns c
def EncryptEMN(e, m, n)
def EncryptEMPQ(e, m, n)

# Returns m
def DecryptCDN(c, d, n)
def DecryptCEPQ(c, e, p, q)
def DecryptCDPQ(c, d, p, q)

# returns string representation of m
def DecodeM(m)

# returns int representation of m
def EncodeM(m)
\end{minted}

\subsubsection{Uses in CTFs:}
\vspace{-4mm}
\begin{tcolorbox}[tab2,tabularx={X|X}]
\href{https://github.com/2kofawsome/CTFwriteups/tree/master/PBJarCTF/ReallynotSecureAlgorithm}{PBJarCTF/\tbf{ReallynotSecureAlgorithm}}     & filler     \\\hline
filler 
\end{tcolorbox}

\newpage

\section{Low Public Exponent Attack}
\vspace{-5mm}
\subsection{Cube-Root Attack}

Given N, e, c where e is very small\\
it is entirely possible that $m^e < N$, thus since $m^e = c \pmod{N}$ its possible that $m$ can be found with the $e^{th}$ root of $c$

This function uses the bisection method to determine the $e^{th}$ root of $c$, starting with a range of $[1, N]$ and halfing each time

Future changes could use Newton's Method, but for now this works

\vspace{-4mm}
\begin{minted}{python}
# returns e root of c, m if m^e < n
def CubeRootCEN(c, e, n)
\end{minted}

\begin{tcolorbox}[tab2,tabularx={X|X}]
\href{https://github.com/2kofawsome/CTFwriteups/tree/master/CSAWCTF/GottaDecryptThemAll}{CSAWCTF/\tbf{GottaDecryptThemAll}}
\end{tcolorbox}



\newpage

\section{Low Private Exponent Attack}
\vspace{-5mm}
\subsection{Wiener's Attack}

More information later


\vspace{-4mm}
\begin{minted}{python}
# Returns d
import owiener
d = owiener.attack(e, N)
\end{minted}

\begin{tcolorbox}[tab2,tabularx={X|X}]

\end{tcolorbox}



\newpage

\section{Close Primes Attack}
\vspace{-5mm}
\subsection{Equal Primes Attack}
\subsection{Sexy-Primes Attack}
\subsection{Twin-Primes Attack}

More information later

\vspace{-4mm}
\begin{minted}{python}
# Returns p, q
upper = N
lower = 1
while True:
    mid = (upper + lower) // 2
    if (mid * mid < N):
        if (lower == mid):
            break
        lower = mid
    else:
        if (upper == mid):
            break
        upper = mid
import sympy
p = sympy.nextprime(lower)
if (p * (p-6) == N):
    q = p - 6
elif (p * (p+6) == N):
    q = p + 6
\end{minted}

\begin{tcolorbox}[tab2,tabularx={X|X}]

\end{tcolorbox}


\newpage

\section{Oracle}
\vspace{-5mm}
\subsection{Leaked LSB Attack}

More information later

\vspace{-4mm}
\begin{minted}{python}
# Returns m as upper
def oracle(c):
    tn.write(str.encode(str(c)) + b"\n")
    response = tn.read_until(b"Would you like to continue? (yes/no)")
    tn.write(b"yes\n")
    tn.read_until(b"What would you like to decrypt? (please respond with an integer)")
    if response == (b'\r\n\r\nThe oracle responds with: 0\r\nWould you like to continue? (yes/no)'):
        return 0
    else:
        return 1
c = (c * c_of_2) % N
k = N.bit_length()
decimal.getcontext().prec = k
lower = decimal.Decimal(0)
upper = decimal.Decimal(N)
for i in range(k):
    possible_plaintext = (lower + upper) / 2

    if not oracle(c):
        upper = possible_plaintext
    else:
        lower = possible_plaintext
    c = (c * c_of_2) % N
\end{minted}

\begin{tcolorbox}[tab2,tabularx={X|X}]

\end{tcolorbox}
\newpage

\section{Divisor Attacks}
\vspace{-5mm}
\subsection{Multiple Factors Attack}
\subsection{Small Primes Factors Attack}

More information later

\vspace{-4mm}
\begin{minted}{python}
e = 65537
n = 50630448182626893495464810670525602771527685838257974610483435332349728792396826591558947027657819590790590829841808151825744184405725893984330719835572507419517069974612006826542638447886105625739026433810851259760829112944769101557865474935245672310638931107468523492780934936765177674292815155262435831801499197874311121773797041186075024766460977392150443756520782067581277504082923534736776769428755807994035936082391356053079235986552374148782993815118221184577434597115748782910244569004818550079464590913826457003648367784164127206743005342001738754989548942975587267990706541155643222851974488533666334645686774107285018775831028090338485586011974337654011592698463713316522811656340001557779270632991105803230612916547576906583473846558419296181503108603192226769399675726201078322763163049259981181392937623116600712403297821389573627700886912737873588300406211047759637045071918185425658854059386338495534747471846997768166929630988406668430381834420429162324755162023168406793544828390933856260762963763336528787421503582319435368755435181752783296341241853932276334886271511786779019664786845658323166852266264286516275919963650402345264649287569303300048733672208950281055894539145902913252578285197293
c = 15640629897212089539145769625632189125456455778939633021487666539864477884226491831177051620671080345905237001384943044362508550274499601386018436774667054082051013986880044122234840762034425906802733285008515019104201964058459074727958015931524254616901569333808897189148422139163755426336008738228206905929505993240834181441728434782721945966055987934053102520300610949003828413057299830995512963516437591775582556040505553674525293788223483574494286570201177694289787659662521910225641898762643794474678297891552856073420478752076393386273627970575228665003851968484998550564390747988844710818619836079384152470450659391941581654509659766292902961171668168368723759124230712832393447719252348647172524453163783833358048230752476923663730556409340711188698221222770394308685941050292404627088273158846156984693358388590950279445736394513497524120008211955634017212917792675498853686681402944487402749561864649175474956913910853930952329280207751998559039169086898605565528308806524495500398924972480453453358088625940892246551961178561037313833306804342494449584581485895266308393917067830433039476096285467849735814999851855709235986958845331235439845410800486470278105793922000390078444089105955677711315740050638

primes = [2293226687, 2444333767, 2602521199, 2695978183, 2724658201, 2753147143, 2772696307, 2824169389, 2841115943, 2944751701, 2949007619, 2959325459,
     3056689019, 3057815377, 3228764447, 3238771411, 3417563069, 3638373857, 3716991893, 3986329331, 4140261491, 4152726959, 4218138251, 2436598001,
     2525697263, 2647129697, 2661720221, 2672301743, 2944722127, 3278196319, 3335574511, 3380851417, 3625437121, 3941016503, 2365186141, 2746638019,
     2963383867, 3013564231, 3464370241, 3646337561, 3760232953, 3978832967, 4006267823, 4235456317, 2223202649, 2278427881, 2388797093, 2682518317,
     2858807113, 3130133681, 3589083991, 3684423151, 3991834969, 2322142411, 2510750149, 2575495753, 2657405087, 2854321391, 3012495907, 3174322859,
     3177943303, 3200434847, 3303691121, 3319529377, 3346647649, 3453863503, 3487902133, 3648309311, 3789253133, 3789746923, 3861767519, 3865448239,
     3943871257, 4045323871, 4198942673, 4227099257, 2733527227, 4270521797, 2371079143, 2424270803, 2572542211, 2636069911, 2752963847, 3083881387,
     3488338697, 3721186793, 3833706949, 4056085883, 2148630611, 2216411683, 2707095227, 3279018511, 3522596999, 3623581037, 3833824031, 3854175641,
     4091945483, 4135004413, 4141964923, 4276173893, 2157385673, 2240170147, 2459187103, 2491570349, 2719924183, 3961738709, 4073647147, 4098491081,
     4205028467, 4252196909, 2230630973, 2703629041, 2710524571, 2932152359, 3035438359, 3180301633, 3285444073, 3291377941, 3359249393, 3398567593,
     3411506629, 3539958743, 3686523713, 3811207403, 3860554891, 3923208001, 3959814431, 3994425601]

def egcd(a, b):
    if a == 0:
        return (b, 0, 1)
    else:
        g, y, x = egcd(b % a, a)
        return (g, x - (b // a) * y, y)

def modinv(a, m):
    g, x, y = egcd(a, m)
    if g != 1:
        raise Exception('modular inverse does not exist')
    else:
        return x % m

ts = []
xs = []
ds = []

for i in range(len(primes)):
	ds.append(modinv(e, primes[i]-1))

m = primes[0]

for i in range(1, len(primes)):
	ts.append(modinv(m, primes[i]))
	m = m * primes[i]

for i in range(len(primes)):
	xs.append(pow((c%primes[i]), ds[i], primes[i]))

x = xs[0]
m = primes[0]

for i in range(1, len(primes)):
	x = x + m * ((xs[i] - x % primes[i]) * (ts[i-1] % primes[i]))
	m = m * primes[i]

print (x%n)
\end{minted}

\begin{tcolorbox}[tab2,tabularx={X|X}]
\href{https://github.com/2kofawsome/CTFwriteups/tree/master/corCTF/4096}{corCTF/\tbf{4096}}
\end{tcolorbox}


















\end{document}